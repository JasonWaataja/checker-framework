\htmlhr
\chapterAndLabel{Third-party checkers\label{external-checkers}}{third-party-checkers}

The Checker Framework has been used to build other checkers that are not
distributed together with the framework.  This chapter mentions just a few
of them.  They are listed in chronological order; older ones appear first
and newer ones appear last.

They are externally-maintained, so if you have problems or questions, you
should contact their maintainers rather than the Checker Framework
maintainers.

If you want this chapter to reference your checker,
please send us a link and a short description.


% Note to maintainers:
% Sections are added to this chapter in chronological order.


\sectionAndLabel{Typestate checkers}{typestate-checker}

In a regular type system, a variable has the same type throughout its
scope.
In a typestate system, a variable's type can change as operations
are performed on it.

The most common example of typestate is for a \<File> object.  Assume a file
can be in two states, \<@Open> and \<@Closed>.  Calling the \<close()> method
changes the file's state.  Any subsequent attempt to read, write, or close
the file will lead to a run-time error.  It would be better for the type
system to warn about such problems, or guarantee their absence, at compile
time.

Just as you can extend the Subtyping Checker to create a type-checker, you can
extend a typestate checker to create a type-checker that supports typestate
analysis.
%% As of April 2020, this URL (first created in Dec 2008) on longer exists.
% An extensible typestate analysis by Adam Warski that builds on
% the Checker Framework is available at
% \myurl{http://www.warski.org/typestate.html}.


\subsectionAndLabel{Comparison to flow-sensitive type refinement}{typestate-vs-type-refinement}

The Checker Framework's flow-sensitive type refinement
(Section~\ref{type-refinement}) implements a form of typestate analysis.
For example, after code that tests a variable against null, the Nullness
Checker (Chapter~\ref{nullness-checker}) treats the variable's type as
\<@NonNull \emph{T}>, for some \<\emph{T}>\@.

For many type systems, flow-sensitive type refinement is sufficient.  But
sometimes, you need full typestate analysis.  This section compares the
two.
% (Dependent types and unused variables
(Unused variables
% (Section~\ref{unused-fields-and-dependent-types})
(Section~\ref{unused-fields})
also have similarities
with typestate analysis and can occasionally substitute for it.  For
brevity, this discussion omits them.)

A typestate analysis is easier for a user to create or extend.
Flow-sensitive type refinement is built into the Checker Framework and is
optionally extended by each checker.  Modifying the rules requires writing
Java code in your checker.  By contrast, it is possible to write a simple
typestate checker declaratively, by writing annotations on the methods
(such as \<close()>) that change a reference's typestate.

A typestate analysis can change a reference's type to something that is not
consistent with its original definition.  For example, suppose that a
programmer decides that the \<@Open> and \<@Closed> qualifiers are
incomparable --- neither is a subtype of the other.  A typestate analysis
can specify that the \<close()> operation converts an \<@Open File> into a
\<@Closed File>.  By contrast, flow-sensitive type refinement can only give
a new type that is a subtype of the declared type --- for flow-sensitive
type refinement to be effective, \<@Closed> would need to be a child of
\<@Open> in the qualifier hierarchy (and \<close()> would need to be
treated specially by the checker).



\sectionAndLabel{Units and dimensions checker}{units-and-dimensions-checker}

A checker for units and dimensions is available at
\url{https://www.lexspoon.org/expannots/}.

Unlike the Units Checker that is distributed with the Checker Framework
(see Section~\ref{units-checker}), this checker includes dynamic checks and
permits annotation arguments that are Java expressions.  This added
flexibility, however, requires that you use a special version both of the
Checker Framework and of the javac compiler.


\sectionAndLabel{Thread locality checker}{loci-thread-locality-checker}

Loci~\cite{WrigstadPMZV2009}, a checker for thread locality, is available at
\url{http://www.it.uu.se/research/upmarc/loci/}.
%% This URL is broken as of
% Developer resources are available at the project page
% \url{http://java.net/projects/loci/}.

% A paper was publishd in ECOOP 2009, release 0.1 was made in March 2011,
% but as of October 2013 and June 2017 the manual is still listed as "forthcoming".


% In a mail from Amanj Mahmud <amanjpro@gmail.com> on 28.03.2011:

% The plugin name:
% ``Loci: A Pluggable Type Checker for Expressing Thread Locality in
% Java''

% Project homepage: http://www.it.uu.se/research/upmarc/loci

% Project's developer's page: http://java.net/projects/loci


\sectionAndLabel{Safety-Critical Java checker}{safety-critical-java-checker}

A checker for Safety-Critical Java (SCJ, JSR 302)~\cite{TangPJ2010} is available at
\url{https://www.cs.purdue.edu/sss/projects/oscj/checker/checker.html}.
Developer resources are available at the project page
\url{https://code.google.com/archive/p/scj-jsr302/}.


% In a mail from Aleš Plšek <aplsek@gmail.com> on 29.03.2011:

% Name: SCJ Checker
% WWW: http://sss.cs.purdue.edu/projects/oscj/checker/checker.html
% Source-Code Repository: http://code.google.com/p/scj-jsr302/

% Description: The SCJ Checker implements verification of a set of
% annotations defined by the Safety-Critical Java standard (JSR-302).
% The checker mainly focuses on proving memory safety of Java programs
% that use a region-based memory management.

% Publications: Static checking of safety critical Java annotations:
% http://portal.acm.org/citation.cfm?doid=1850771.1850792


\sectionAndLabel{Generic Universe Types checker}{gut-checker}

A checker for Generic Universe Types~\cite{DietlEM2011}, a lightweight ownership type
system, is available from
\url{https://ece.uwaterloo.ca/~wdietl/ownership/}.


\sectionAndLabel{EnerJ checker}{enerj-checker}

A checker for EnerJ~\cite{SampsonDFGCG2011}, an extension to Java that exposes hardware faults
in a safe, principled manner to save energy with only
slight sacrifices to the quality of service, is available from
\url{http://sampa.cs.washington.edu/research/approximation/enerj.html}.


\sectionAndLabel{CheckLT taint checker}{checklt-checker}

CheckLT uses taint tracking to detect illegal information flows, such as
unsanitized data that could result in a SQL injection attack.
CheckLT is available from \url{http://checklt.github.io/}.


\sectionAndLabel{SPARTA information flow type-checker for Android}{sparta-checker}

SPARTA is a security toolset aimed at preventing malware from appearing in
an app store.  SPARTA provides an information-flow type-checker that is
customized to Android but can also be applied to other domains.
The SPARTA toolset is available from
\url{https://checkerframework.org/sparta/}.
The paper
\href{https://homes.cs.washington.edu/~mernst/pubs/infoflow-ccs2014.pdf}{``Collaborative
    verification of information flow for a high-assurance app store''}
  appeared in CCS 2014.


\sectionAndLabel{Immutability checkers:  IGJ, OIGJ, and Javari\label{javari-checker}}{igj-checker}

Javari~\cite{TschantzE2005}, IGJ~\cite{ZibinPAAKE2007}, and
OIGJ~\cite{ZibinPLAE2010} are type systems that enforce immutability
constraints.  Type-checkers for all three type systems were distributed
with the Checker Framework through release 1.9.13 (dated 1 April 2016).
If you wish to use them, install
\href{https://checkerframework.org/releases/1.9.13/}{Checker
  Framework version 1.9.13}.

They were removed from the main distribution on June 1, 2016 because the
implementations were not being maintained as the Checker Framework evolved.
The type systems are valuable, and some people found the type-checkers
useful.  However,
% the type-checkers should be rewritten from scratch and
we wanted
to focus on distributing checkers that are currently being maintained.


\sectionAndLabel{Read Checker for CERT FIO08-J}{read-checker}

CERT
rule \href{https://www.securecoding.cert.org/confluence/display/java/FIO08-J.+Distinguish+between+characters+or+bytes+read+from+a+stream+and+-1}{FIO08-J}
describes a rule for the correct handling of characters/bytes read
from a stream.

The Read Checker enforces this rule.
It is available from
\url{https://github.com/opprop/ReadChecker}.


\sectionAndLabel{SQL checker that supports multiple dialects}{sql-schecker}

\href{http://www.jooq.org/}{jOOQ} is a database API that lets you build
typesafe SQL queries.  jOOQ version 3.0.9 and later ships with a SQL
checker that provides even more safety:  it ensures that you don't
use SQL features that are not supported by your database
implementation.  You can learn about the SQL checker at
\url{https://blog.jooq.org/2016/05/09/jsr-308-and-the-checker-framework-add-even-more-typesafety-to-jooq-3-9/}.


\sectionAndLabel{Glacier:  Class immutability}{glacier-immutability-checker}

\href{http://mcoblenz.github.io/Glacier/}{Glacier}~\cite{CoblenzNAMS2017}
enforces transitive class immutability in Java.  According to its webpage:

\begin{itemize}
\item
  Transitive: if a class is immutable, then every field must be
  immutable. This means that all reachable state from an immutable object's
  fields is immutable.
\item
  Class: the immutability of an object depends only on its class's
  immutability declaration.
\item
  Immutability: state in an object is not changable through any reference to
  the object.
\end{itemize}


\sectionAndLabel{UI Thread Checker for ReactiveX}{rx-thread-checker}

The \href{https://plv.colorado.edu/benno/ase18.pdf}{Rx Thread \& Effect Checker}~\cite{SteinCSC2018} enforces
UI Thread safety properties for stream-based Android applications and is available at
\url{https://github.com/uber-research/RxThreadEffectChecker}.


\sectionAndLabel{AWS KMS compliance checker}{kms-compliance-checker}

The \href{https://github.com/awslabs/aws-kms-compliance-checker}{AWS KMS
  compliance checker} extends the Constant Value Checker (see
\chapterpageref{constant-value-checker}) to enforce that calls to Amazon
Web Services' Key Management System only request 256-bit (or longer) data
keys.  This checker can be used to help enforce a compliance requirement
(such as from SOC or PCI-DSS) that customer data is always encrypted with
256-bit keys.


\sectionAndLabel{AWS crypto policy compliance checker}{crypto-policy-compliance-checker}

The
\href{https://github.com/awslabs/aws-crypto-policy-compliance-checker}{AWS
  crypto policy compliance checker} checks that no weak cipher algorithms
are used with the Java crypto API\@.


% LocalWords:  SCJ EnerJ CheckLT unsanitized JavaUI CCS IGJ OIGJ FIO08
%%  LocalWords:  jOOQ typesafe
