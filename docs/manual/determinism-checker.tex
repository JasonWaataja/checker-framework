\htmlhr
\chapter{Determinism Checker\label{determinism-checker}}

A nondeterministic program is one that may produce different output when
it is run twice with the same input on the same machine.
It is desirable for programs to be deterministic.
It is easier to test a deterministic program, because nondeterminism can
lead to flaky tests that sometimes succeed and sometimes fail.
It is easier to reproduce a bug in a deterministic program.
It is easier for a user or programmer to compare two
deterministic executions than two nondeterministic executions.

The Determinism Checker helps you make your programs deterministic or
recognize when they are not.
The Determinism Checker enables a programmer to indicate which
computations should be the same across runs of a single-threaded program,
and then the Determinism Checker verifies that property.

Even single-threaded programs are frequently nondeterministic.
One cause is use of the \<Random>, \<Date>, and \<Time> classes.
Another cause is use of hash tables (including \<HashSet>s and
\<HashMap>s), whose iteration order is nondeterministic across runs.
For example, even if the set contains the same values, the following code
may produce output in different orders:

\begin{Verbatim}
  for (Object x : mySet) {
    System.out.println(x);
  }
\end{Verbatim}

Another common cause of nondeterminism is the operating system scheduler,
for concurrent programs.  The Determinism Checker does not address this
cause of nondeterminism.  Also, the Determinism Checker does not warn
about the possibility that the file system contains different files or
contents on different runs of a program, or that system properties (such as
line and path separators) vary across different machines.

To run the Determinism Checker, supply the
\code{-processor DeterminismChecker}
command-line option to javac.
For example:

\begin{alltt}
  javac -processor DeterminismChecker InputFile.java
\end{alltt}

%./bin/javac -cp NonDetTypeChecker/jsr308/checker-framework/checker/build -processor org.checkerframework.checker.determinism.DeterminismChecker NDCheckerTest/src/TestListUnsafe2.java


\section{Determinism qualifiers\label{determinism-qualifiers}}

The Determinism type system uses the following type qualifiers (see Figure~\ref{fig-determinism-hierarchy}):
\begin{description}
\item[\refqualclass{checker/determinism/qual}{NonDet}] indicates
  that the expression might have different values in two different executions on the same machine.
\item[\refqualclass{checker/determinism/qual}{OrderNonDet}] indicates that
  a collection (i.e any class that is a subtype of java.util.Collection or java.util.Iterator) or an array will have the same elements in every execution, but in a
  possibly different order.  \<@OrderNonDet> may only be written on
  collections and arrays.
\item[\refqualclass{checker/determinism/qual}{Det}] indicates that
  the expression evaluates to the same value (with respect to \<.equals()>) in all
  executions on the same machine; for a collection, iteration also yields the values in the same
  order.
  This is the default qualifier.
\end{description}

Section~\ref{determinism-polymorphism} explains polymorphic qualifiers.

\begin{figure}
  \begin{center}
    \includeimage{determinism}{4cm}
  \end{center}
\caption{Determinism type hierarchy}
\label{fig-determinism-hierarchy}
\end{figure}


\section{Valid collection and array types\label{determinism-collection-types}}

The (determinism) type of a Collection or Iterator must be a supertype or equal to
the type of the type parameter (see Figure~\ref{fig-determinism-collections}).

For example, these types are valid:
\begin{itemize}
    \item \code{@NonDet\ \ \ \ \ \ List<@NonDet\ \ \ \ \ \ Integer>}
    \item \code{@NonDet\ \ \ \ \ \ List<@OrderNonDet Set<...>\relax >}
    \item \code{@NonDet\ \ \ \ \ \ List<@Det\ \ \ \ \ \ \ \ \ Integer>}
    \item \code{@Det\ \ \ \ \ \ \ \ \ List<@Det\ \ \ \ \ \ \ \ \ Integer>}
    \item \code{@OrderNonDet Set <@OrderNonDet Set<...>\relax >}
    \item \code{@OrderNonDet Set <@Det String>}
    \item \code{@NonDet Set \ <T extends @NonDet Object>}
    \item \code{@NonDet Set \ <T extends @Det Object>}
\end{itemize}

These types are invalid:
  \begin{itemize}
    \item \code{@OrderNonDet Set <@NonDet\ \ \ \ \ \ Integer>}
    \item \code{@Det\ \ \ \ \ \ \ \ \ List<@NonDet\ \ \ \ \ \ Integer>}
    \item \code{@Det\ \ \ \ \ \ \ \ \ List<@OrderNonDet Set<...>\relax >}
    \item \code{@Det\ \ \ \ \ \ \ \ \ List<T extends @NonDet Object>}
  \end{itemize}

Similarly, the (determinism) type of an array must be a supertype or equal to
the type of the component type (see Figure~\ref{fig-determinism-collections}).

For example, these types are valid:
\begin{itemize}
    \item \code{@NonDet int @NonDet \ \ \ \ []}
    \item \code{@NonDet int @OrderNonDet[]}
    \item \code{@Det \ \ \ int @NonDet \ \ \ \ []}
    \item \code{@Det \ \ \ int @Det \ \ \ \ \ \ \ []}
    \item \code{@Det \ \ \ int @OrderNonDet[] @OrderNonDet[]}
\end{itemize}

These types are invalid:
\begin{itemize}
    \item \code{@NonDet int @OrderNonDet[]}
    \item \code{@NonDet int @Det \ \ \ \ \ \ \ []}
\end{itemize}

\begin{figure}
  \centering
  \begin{tabular}{|l|l|l|l|l|}
    \cline{3-5}
    \multicolumn{2}{c|}{~}  &  \multicolumn{3}{c|}{Type Argument (or array component type)} \\ \cline{3-5}
    \multicolumn{2}{c|}{~}  & NonDet     & OrderNonDet & Det \\ \hline
              & NonDet      &   valid    &  valid      & valid  \\ \cline{2-5}
Collection (or array)   & OrderNonDet &   invalid  &  valid      & valid  \\ \cline{2-5}
              & Det         &   invalid  &  invalid    & invalid      \\ \hline
  \end{tabular}
  \caption{Valid Collection (and array) declarations.  The Collection's (or array's) type qualifier
    must be a supertype or equal to the type argument (or array component type).}
  \label{fig-determinism-collections}
\end{figure}

\subsection{Valid user defined types\label{determinism-valid-user-types}}
A user defined type may be annotated as \<@OrderNonDet> if and only if it
is a subtype of Collection or Iterator.
For example, if a user defines a type as
\begin{Verbatim}
    public class TestUserCollection<E> extends ArrayList<E> {...}
\end{Verbatim}
Writing \code{@OrderNonDet TestUserCollection<@Det Integer>} is valid.\\
Writing \code{@OrderNonDet TestUserCollection<@NonDet Integer>} is invalid
because the type of the type parameter (\code{@NonDet}) is not a subtype
of the user defined Collection type (\code{@OrderNonDet}).

User defined types that are not a subtype of Collection or Iterator
may not be annotated as \code{@OrderNonDet}.
%Formally, the type validity rule is stated as follows:

%\begin{prooftree}
%  \AxiomC{is\_Collection\_Type(tc)}
%  \AxiomC{outer\_type(tx) <= tc}
%  \BinaryInfC{|- : tc<tx> }
%\end{prooftree}

% \begin{verbatim}
% Type outer_type( Type tx )
% {
%     if ( ! is_Generic_type(tx) )
%         return tx
%     Assert (tx == ty<tz>)
%         return ty
% }
% \end{verbatim}


\section{Controlling polymorphism\label{determinism-polymorphism}}

\begin{description}
\item[\refqualclass{checker/determinism/qual}{PolyDet}] is a
  polymorphic qualifier over nondeterminism (see Section~\ref{qualifier-polymorphism}).
\end{description}

The Determinism Checker makes
several enhancements to standard polymorphic qualifiers.


\subsection{Avoiding @OrderNonDet\label{determinism-polymorphism-ordernondet}}

\begin{description}
\item[\refqualclass{checker/determinism/qual}{PolyDet}\<("down")>]
  Acts like \<@Det> when the polymorphic qualifier is instantiated as
  \<@OrderNonDet>.
  ``\<down>'' refers to the direction in the qualifier hierarchy:
  \<@OrderNonDet> is changed to a lower qualifier that is a subtype of it.
\item[\refqualclass{checker/determinism/qual}{PolyDet}\<("up")>]
  Acts like \<@NonDet> when the polymorphic qualifier is instantiated as
  \<@OrderNonDet>.
  ``\<up>'' refers to the direction in the qualifier hierarchy:
  \<@OrderNonDet> is changed to a higher qualifier that is a supertype of it.
\end{description}

Ordinarily, every occurrence of \<@PolyDet> is instantiated to the same
type.  Sometimes this does not capture the desired behavior.  For instance,
there should be three versions of the \<List.add> and \<List.hashCode> routines:

\begin{Verbatim}
         @NonDet boolean add(      @NonDet List<E> this, E e)
            @Det boolean add( @OrderNonDet List<E> this, E e)
            @Det boolean add(         @Det List<E> this, E e)

         @NonDet int hashCode(     @NonDet List<E> this)
         @NonDet int hashCode(@OrderNonDet List<E> this)
            @Det int hashCode(        @Det List<E> this)
\end{Verbatim}

A standard polymorphic annotation

\begin{Verbatim}
        @PolyDet boolean add(     @PolyDet List<E> this, E e)
        @PolyDet int hashCode(    @PolyDet List<E> this)
\end{Verbatim}

\noindent
can express the first and third versions of each routine, but not the
second one.  The \refqualclass{checker/determinism/qual}{PolyDet}\<("down")> and
\refqualclass{checker/determinism/qual}{PolyDet}\<("up")> annotations enable
expressing the method behavior:

\begin{Verbatim}
@PolyDet("down") boolean add(     @PolyDet List<E> this, E e)
  @PolyDet("up") int hashCode(    @PolyDet List<E> this)
\end{Verbatim}

See the annotated JDK for more example uses of the annotations.

Since the type \<@OrderNonDet> is invalid for non-collections,
if \<@PolyDet> is applied to a collection somewhere in the method signature
and if the return type isn't a collection, then the return type should be
annotated as \<@PolyDet("up")> or \<@PolyDet("down")>.

\subsubsection{Special handling of Set.equals()\label{determinism-improved-precision-set-equals}}

In the JDK, the \<equals()> method of \<Set> is annotated as
\begin{Verbatim}
@PolyDet("up") boolean equals(@PolyDet Set<E> this, @PolyDet Object o);
\end{Verbatim}
While this is sound, it is imprecise.
For example, consider the code below:

\begin{Verbatim}
    @OrderNonDet Set<@Det Integer> a;
    @OrderNonDet Set<@Det Integer> b;
    a.equals(b);
\end{Verbatim}

Here, the return type of \<equals()> gets the annotation \<@NonDet>.
But it would be more precise to annotate it as \<@Det>.
To achieve this precision, the checker annotates the return type
of \<equals()> called on a receiver of type \<Set> as \<@Det> if the
following conditions are true:
\begin{itemize}
    \item the receiver does not have {@code List} or its subtype as a type parameter,
    \item the argument to equals() is also an {@code @OrderNonDet Set}, and
    \item the argument to equals() also does not have {@code List} or its subtype as a
    type parameter
\end{itemize}

\subsection{Distinguishing bindings and uses\label{determinism-polymorphism-binding}}

\begin{description}
\item[\refqualclass{checker/determinism/qual}{PolyDet}\<("use")>]
  Has the same type as any \<@PolyDet> annotations in the method
  declaration, but does not affect their instantiation.
\end{description}

As explained in Section~\ref{qualifier-polymorphism},
all of the \<@PolyDet> annotations in a method declaration are given the same
instantiation, which is one that works for all of the actual arguments
(their least upper bound).

The \<@PolyDet("use")> annotation is given the same instantiation as all of the
\<@PolyDet> annotations, but does not affect the instantiation. This is especially
useful in preventing methods from non-deterministically modifying the
state of a deterministic receiver.

As an example of the need for the \<@PolyDet("use")> annotation, consider the
\<List.set> method.  There should be two versions of it (\<@OrderNonDet
int> is not a valid type, so no \<@OrderNonDet> instantiation is created):

\begin{Verbatim}
E set(@NonDet List<E> this, @NonDet int index, E element)
E set(   @Det List<E> this,    @Det int index, E element)
\end{Verbatim}

\noindent
However, it should \emph{not} be legal for a client to make a call such as
the following:

\begin{Verbatim}
@Det List<...> myDetList;
@NonDet int random;
myDetList.set(random, ...);
\end{Verbatim}

\noindent
The type system must forbid this call, because it modifies a
\<@Det> list in a nondeterministic way.
However, the arguments are legal for the first instantiation:
each argument type is a subtype of the corresponding declared type.

To solve this problem, the instantiation is chosen based only on the first
argument (the \<List> argument).  To have this effect, the library method
is annotated as:

\begin{Verbatim}
E set(@PolyDet List<E> this, @PolyDet("use") int index, E element)
\end{Verbatim}



\section{How to customize the Determinism Checker\label{determinism-customization}}

In the JDK, printing methods such as \<println> are annotated as taking a
deterministic result.  This guarantees that a program's output will be
deterministic.  If it is be too restrictive for your program:
\begin{itemize}
\item
  You can define your own printing methods, some of which can take
  nondeterministic inputs.
\item
  You can override the JDK annotations by providing a stub file
  (Section~\ref{stub}).
\end{itemize}


\section{Default annotations\label{determinism-defaults}}

The default annotation for most unannotated types is \<@Det>.

\subsection{Method signatures\label{determinism-method-defaults}}
\begin{itemize}
\item
  Formal parameter and return types default to \<@PolyDet>.
\item
  If a formal parameter/return type is an array, its component type defaults to
  \<@PolyDet>.
\item
  As an exception, the return type of a method with no unannotated or
  \<@PolyDet> formal parameters (including the receiver) defaults to
  \<@Det>.
\end{itemize}

These choices permit all possible arguments to an unannotated method.

\subsection{Constructor signatures\label{determinism-constructor-defaults}}
The default annotation on constructor parameters and return types is also \<@PolyDet>.
As a result, one can construct Objects whose type
is dependent on the types of constructor arguments without any explicit
annotations. This causes the checker
to report errors in situations where the constructor modifies the fields of its class
using these arguments. The reason for this is that the fields default to \<@Det>.
As an illustration, consider the example below:

\begin{Verbatim}
    public class StringHolder {
        public String data;
        public StringHolder(String data) {
            this.data = data;
        }
    }
\end{Verbatim}

In this example, the checker reports an error "assignment type incompatible"
at the line \<this.data = data> because the field \<data> is annotated as
\<@Det> by default, whereas the constructor argument has \<@PolyDet>
as its default annotation. In such cases, you could do either of the following:
\begin{itemize}
    \item Suppress the warning in the constructor (at line \<this.data = data> in the above example)
    to allow construction of \<@PolyDet> objects.
    \item Explicitly annotate the constructor arguments and return types. For example, to allow
    the construction of only deterministic \<StringHolder> objects, annotate the above code as
    \begin{Verbatim}
        public class StringHolder {
            public String data;
            public @Det StringHolder(@Det String data) {
            this.data = data;
            }
        }
    \end{Verbatim}
\end{itemize}

\subsection{Local array and generic types\label{determinism-array-defaults}}
As with all type systems, defaulting is different for the array and
component types in a locally-declared array or generic type without explicit type
annotations.
The array type defaulting follows the CLIMB to top rule (see Section~\ref{climb-to-top}) and
the component type defaulting uses the default type qualifier (\<@Det>, for
the Determinism Checker).
For example, the local array declaration \<int[] a;> is defaulted to \<@Det int @NonDet [] a;>.
The local declaration \code{MyClass<MyType> x;} is defaulted to
\code{@NonDet MyClass<@Det MyType> x;}.

\section{Implicit annotations\label{determinism-implicits}}
The parameter type of the \code{main} method has an implicit annotation
\<@Det String @Det[]>.

%%  LocalWords:  DeterminismChecker HashSet InputFile NonDet Det MyClass
%%  LocalWords:  OrderNonDet PolyDet UsePolyDet println MyType
