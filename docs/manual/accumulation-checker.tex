\htmlhr
\chapterAndLabel{Building an accumulation checker}{accumulation-checker}

%% This chapter should appear after the "creating a checker" chapter, or perhaps as part of it,
%% once accumulation support is complete.

This chapter describes how to build a checker for an accumulation analysis.
If you want to \emph{use} an existing checker, you do not need to read this chapter.

An \emph{accumulation analysis} is a program analysis where the
analysis abstraction is a monotonically increasing set --- that is, the
analysis learns new facts, and facts are never retracted.
Typically, some operation in code is legal
only when the set is large enough --- that is, the estimate has accumulated
sufficiently many facts.

Accumulation analysis is a special case of typestate analysis in which
(1) the order in which operations are performed does not affect what is subsequently legal,
and (2) the accumulation does not add restrictions; that is, as
more operations are performed, more operations become legal.
Unlike a traditional typestate analysis, an accumulation analysis does
not require an alias analysis for soundness. It can therefore be implemented
as a flow-sensitive type system.

The Checker Framework contains a generic accumulation analysis that you can
extend to create your own accumulation checker.

Before reading the rest of this chapter, you should
read how to create a checker (Chapter~\ref{creating-a-checker}).
The rest of this chapter assumes you have done so.


\paragraphAndLabel{Defining type qualifiers}{accumulation-qualifiers}

The Object Construction Checker (Section~\ref{object-construction-checker})
is a concrete example of a practical accumulation checker that you may want
to reference. The
\href{https://github.com/typetools/checker-framework/blob/master/framework/src/test/java/testaccumulation/TestAccumulationChecker.java}{Test Accumulation Checker}
is a simplified version of the same that you can also use as a model.

First, design your analysis.  Decide what
your checker will accumulate and how to represent it. The Checker Framework's
support for accumulation analysis requires you to accumulate a string representation
of whatever you are accumulating. For example, when accumulating which methods have
been called on an object, you might choose to accumulate method names.

Define a type qualifier that has a single argument: a \<String[]> named \<value>.
The default value for this argument should be an empty array. The
qualifier should have no supertypes (\<@SubtypeOf({})>) and should
be the default qualifier in the hierarchy (\<@DefaultQualifierInHierarchy>).
An example of such a qualifier can be found in the Checker Framework's tests:
\href{https://github.com/typetools/checker-framework/blob/master/framework/src/test/java/testaccumulation/qual/TestAccumulation.java}{TestAccumulation.java}.

Also define a bottom type, analogous to
\href{https://github.com/typetools/checker-framework/blob/master/framework/src/test/java/testaccumulation/qual/TestAccumulationBottom.java}{TestAccumulationBottom.java}.
It should take no arguments, and should be a subtype of the accumulator type you defined earlier.

\paragraphAndLabel{Setting up the checker}{accumulation-setup}

Define a new class that extends \refclass{common/accumulation}{AccumulationChecker}.
It does not need any content.

Define a new class that extends \refclass{common/accumulation}{AccumulationAnnotatedTypeFactory}.
You must create a new constructor whose only argument is a \refclass{common/basetype}{BaseTypeChecker}.
Your constructor should call the \<super> constructor defined in
\refclass{common/accumulation}{AccumulationAnnotatedTypeFactory}.


\paragraphAndLabel{Adding accumulation logic}{accumulation-accumulating}

Define a class that extends \refclass{common/accumulation}{AccumulationTransfer}.
To update the estimate of what has been accumulated, override
\refclass{framework/flow}{CFAbstractTransfer} to call
\refmethodterse{common/accumulation}{AccumulationTransfer}{accumulate}{-org.checkerframework.dataflow.cfg.node.Node-org.checkerframework.dataflow.analysis.TransferResult-java.lang.String...-}.

For example, when accumulating the names of methods called, you would override
\refmethod{framework/flow}{CFAbstractTransfer}{visitMethodInvocation}{-org.checkerframework.dataflow.cfg.node.MethodInvocationNode-org.checkerframework.dataflow.analysis.TransferInput-},
call \<super> to get a \<TransferResult>, compute the method name from the \<MethodInvocationNode>,
and then call \<accumulate>.
