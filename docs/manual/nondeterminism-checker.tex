\htmlhr
\chapter{Nondeterminism Checker\label{nondeterminism-checker}}

A nondeterministic program is one that may produce different output when
run twice on the same input.  Nondeterministic programs are hard to debug
(it may be difficult to reproduce an incorrect behavior), and it is hard to
compare multiple outputs or logs.

Even single-threaded programs are frequently nondeterministic.
One cause is calls to Random, Date, and Time.
Another cause is use of hash tables, whose iteration order is
nondeterministic across runs.
For example, even if the set contains the same values, the following code
may produce output in different orders:

\begin{Verbatim}
  printSet(HashSet arg) {
    for (Object x : arg) {
      System.out.println(x);
    }
  }
\end{Verbatim}

The Nondeterminism Checker enables a programmer to indicate which
computations should be the same across runs of a single-threaded program,
and then the Nondeterminism Checker verifies that property.

Another common cause of nondeterminism is the operating system scheduler,
for concurrent programs.  The Nondeterminism Checker does not address this
cause of nondeterminism.  Also, the Nondeterminism Checker does not warn
about the possibility that the file system contains different files or
contents on different runs of a program.

To run the Nondeterminism Checker, supply the
\code{-processor NondeterminismChecker}
command-line option to javac.
For example:

\begin{alltt}
	javac -processor NondeterminismChecker inputFile.java
\end{alltt}

%./bin/javac -cp NonDetTypeChecker/jsr308/checker-framework/checker/build -processor org.checkerframework.checker.nondeterminism.NondeterminismChecker NDCheckerTest/src/TestListUnsafe2.java

\section{Nondeterminism annotations for collections\label{nondeterminism-annotations-collections}}

The Nondeterminism type system uses the following annotations:
\begin{itemize}
\item
  \refqualclass{checker/nondeterminism/qual}{ValueNonDet} indicates
  that the collection can have possibly different values in two different executions.
\item
  \refqualclass{checker/nondeterminism/qual}{OrderNonDet} indicates that
  the collection will have the same elements, but in a possibly different order in two different executions.
  \code{@ValueNonDet} is a supertype of \code{@OrderNonDet}.
 \item
  \refqualclass{checker/nondeterminism/qual}{Det} indicates that
  the collection will have the same elements, in the same order across all executions.
  \code{@OrderNonDet} is a supertype of \code{@Det}.
   It is the default qualifier.
\item
  \refqualclass{checker/nondeterminism/qual}{PolyDet} is a
  polymorphic qualifier over nondeterminism (see Section~\ref{qualifier-polymorphism}).
\end{itemize}

\section{Nondeterminism annotations for all other types\label{nondeterminism-annotations-elements}}

\begin{itemize}
\item
  \refqualclass{checker/nondeterminism/qual}{ValueNonDet} indicates
  that this type can possibly have different values across different executions.
 \item
  \refqualclass{checker/nondeterminism/qual}{Det} indicates that
  the value remains the same in any execution.
\end{itemize}

Note that non-collections may not be annotated as \code{@OrderNonDet}.
Figure~\ref{fig-hierarchy} shows the relationship among these
type qualifiers.

\begin{figure}[h]\caption{Type hierarchy}
	\begin{center}
		\includeimage{nondet-hierarchy}{8cm}
	\end{center}
	\label{fig-hierarchy}
\end{figure}

\section{Valid annotations on Collections\label{type-validity}}
The annotations on Collections must satisfy the constraint that the annotation on type parameter of a Collection must be a
subtype of the annotation on the Collection itself. Here are a few examples illustrating this.

\subsection{Examples\label{validity-examples}}
\begin{figure}[h]\caption{Valid types}
	\begin{enumerate}
		\item @Det List<@Det Integer>
		\item @ValueNonDet List<@OrderNonDet Set<@Det String>>
	\end{enumerate}
\end{figure}

\begin{figure}[h]\caption{Invalid types}
	\begin{enumerate}
		\item @Det Set<@ValueNonDetDet Integer> : The @Det annotation on Set means that iterating over its elements will produce the same
		elements in the exact same order across executions. But the @ValueNonDet annotation on Integer violates this guarantee. Therefore, this is an invalid type annotation.
		\item @ValueNonDet List<@OrderNonDet String> - This is invalid because @OrderNonDet can be used only with Collections but it is used with a String here.
	\end{enumerate}
\end{figure}

Table~\ref{Tab:validity} shows the valid annotations on Collections and their type parameters.
The Cells marked XXX are the invalid ones.
\begin{table}[h]
	\centering
	\caption{Valid Collections}
	\label{Tab:validity}
	\begin{tabular}{|l|l|l|l|l|}
		\hline
		& 		\multicolumn{4}{l|}{Type Parameter}		                                                                   \\ \hline
		&             & ValueNonDet                                     & OrderNonDet              & Det \\ \cline{2-5}
		Collection    & ValueNonDet &                                                 &                          &     \\ \cline{2-5}
		& OrderNonDet &   XXX  &                          &     \\ \cline{2-5}
		& Det         &         XXX               &  	XXX	&     \\ \hline
	\end{tabular}
\end{table}

%Formally, the type validity rule is stated as follows:

%\begin{prooftree}
%	\AxiomC{is\_Collection\_Type(tc)}
%	\AxiomC{outer\_type(tx) <= tc}
%	\BinaryInfC{|- : tc<tx> }
%\end{prooftree}

\begin{verbatim}
Type outer\_type( Type tx )
{
    if ( ! is\_Generic\_type(tx) )
        return tx
    Assert (tx == ty<tz>)
        return ty
}
\end{verbatim}

\section{Method annotations\label{method-annotations}}
In addition to annotating the return type and the parameters of a method, users can supply the type annotations -
@Ond2D or @Ond2Vnd on method declarations.
If a return type is annotated as @Polydet and if this @Polydet gets resolved to @OrderNonDet, and if the method declaration has a type annotation @Ond2D,
the checker replaces the return type from @OrderNonDet to @Det. Similarly, when the checker sees a type annotation @Ond2Vnd, it
replaces @OrderNonDet with @ValueNonDet. The next section illustrates how these annotations are used.

\section{Library Annotations\label{library-Annotations}}
\subsection{Methods common to Lists and Sets\label{lib-methods}}
\begin{description}
	\item[@Ond2D @PolyDet  boolean add (@PolyDet List <E> this, E e)] Suppose the List is of
	type @OrderNonDet List<@Det Integer>, the element to be added (E e) must be a subtype of @Det.
	The return type in this case will be annotated as @Det because of the @Ond2D annotation.
	\item[@Ond2D @PolyDet  boolean contains (@PolyDet List <E> this, E e)]
	\item[@Ond2D @PolyDet int size (@PolyDet List <E> this)]
\end{description}

\section{Type system guarantess\label{type-system-guarantees}}
The underlying type system provides the following guarantees:
\begin{enumerate}
	\item Assignment from a subtype to a supertype is flagged as an error.
	\item The formal arguments to a method must be a subtypes of parameters in the declaration. Otherwise, the checker framework reports an error.
\end{enumerate}

\section{Checker guarantees\label{error-reporting}}
In addition to its usual type mismatch errors, the Nondeterminism checker reports the following:
\begin{enumerate}
	\item A \code{print} statement flags an error if any of its arguments has the type annotation @ValueNonDet or @OrderNonDet.
\end{enumerate}

%%  LocalWords:  NondeterminismChecker
