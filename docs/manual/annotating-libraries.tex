\htmlhr
\chapterAndLabel{Annotating libraries}{annotating-libraries}

When your code uses a library that is not currently being compiled,
the Checker Framework looks up the library's annotations in its class files.
Section~\ref{annotated-libraries-using} tells you how to find and use a
version of a library that contains type annotations.

If your code uses a library that does \emph{not} contain type annotations,
then the type-checker has no way to know the library's behavior.
The type-checker
makes conservative assumptions about unannotated bytecode.
% :  it assumes that every method parameter has the bottom type annotation
% and that every method return type has the top type annotation
(See
Section~\ref{defaults-classfile} for details, an example, and how to
override this conservative behavior.)
These conservative library
annotations invariably lead to checker warnings.

This chapter describes how to eliminate
the warnings by adding annotations to the library.
(Alternately, you can instead
suppress all warnings related to an unannotated library, or to part of your
codebase, by use of the
\code{-AskipUses} or \code{-AonlyUses} command-line option; see
Sections~\ref{askipuses}--\ref{askipdefs}.)

You can write annotations for a library, and make them known to a checker, in two ways.

\begin{enumerate}
\item
  Write annotations in a copy of the library's source code (for instance,
  in a fork of the library's GitHub project).  In addition to writing
  annotations, adjust the build system to run pluggable-type-checking when
  compiling (see \chapterpageref{external-tools}).  Now, when you compile
  the library, the resulting \<.jar> file contains type annotations.

  When checking a client of the library,
  put the annotated library's \<.jar> file on the classpath, as explained in
  Section~\ref{annotated-libraries-using}.
  %% You only need to do this if for some reason you don't trust the
  %% artifacts on Maven Central; but don't cater to that level of paranoia
  %% by mentioning it.
  % When \emph{running} your code, you can use either version of the library:  the
  % one you created or the original distributed version.

  %% There is no point to advertising this deprecated workflow.
  % You can insert annotations in the compiled \code{.class} files of the
  % library.  You would express the annotations textually, typically as an
  % annotation index file, and
  % then insert them in the library by using the Annotation File Utilities
  % (\myurl{https://checkerframework.org/annotation-file-utilities/}).
  % See the Annotation File Utilities documentation for full details.

  With this compilation approach, the syntax of the library annotations is
  validated ahead of time.  Thus, this compilation approach is less
  error-prone, and the type-checker runs faster.  You get
  correctness guarantees about the library in addition to your code.

  For instructions, see Section~\ref{annotating-libraries-create}.

\item
  Write annotations in a ``stub file'', if you do not have access to the
  source code.
  %% Leave out this complication.
  % This approach is possible with annotated source code.

  Then, when check a client of the library,
  supply the ``stub file'' textually to the Checker Framework.

  This approach does not require you to compile the library source
  code.
  A stub file is applicable to multiple versions of a library, so
  the stub file does not need to be updated when a new version of the
  library is released, unless the API has changed (such as defining a new
  method).

  For instructions, see Section~\ref{stub}.

\end{enumerate}


If you annotate a new library (either in its source code or in a stub
file), please inform the Checker Framework
developers so that your annotated library can be distributed with the
Checker Framework.
Sharing your annotations is useful even if the library is only partially
annotated.
However, as noted in Sections~\ref{get-started-with-legacy-code} and~\ref{library-tips-fully-annotate}, you
should annotate an entire class at a time.
You may find type inference tools (\chapterpageref{type-inference}) helpful
when getting started, but you should always examine their results.


\sectionAndLabel{Tips for annotating a library}{library-tips}

Section~\ref{tips-about-writing-annotations} gives general tips for writing
annotations.  This section gives tips that are specific to annotating a
third-party library.


\subsectionAndLabel{Don't change the code}{library-tips-dont-change-the-code}

If you annotate a library that you maintain, you can refactor it to improve
its design.

When you annotate a library that you do not maintain, you should only add
annotations and, when necessary, documentation of those annotations.  You
can place the documentation in a Java comment (\<//> or \</*...*/>) or in a
\refqualclass{framework/qual}{CFComment} annotation.

Do not change the library's code, which will change its behavior and make
the annotated version inconsistent with the unannotated version.
(Sometimes it is acceptable to make a refactoring, such as extracting an
expression into a new local variable in order to annotate its type or
suppress a warning.  Perform refactoring only when you cannot use
\<@AssumeAssertion>.)
% (The only time it is acceptable to comment out existing code in the library
% is if the regular Java compiler cannot compile the code; but this should be
% extremely rare.)

Do not change publicly-visible documentation, such as Javadoc comments.
That also makes the annotated version inconsistent with the unannotated
version.

Do not change formatting and whitespace.

All of these changes increase the difference between upstream (the original
version) and your annotated version.  This makes it harder for others to
understand what you have done, and they make it harder to pull changes from
upstream into the annotated library.


\subsectionAndLabel{Library annotations should reflect the specification, not the implementation}{library-tips-specification}

Publicly-visible annotations (including those on public method formal
parameters and return types) should be based on the
documentation, typically the method's Javadoc.
In other words, your annotations should re-state facts that are in the Javadoc
documentation.

Do not add requirements or guarantees beyond what the library author has
already committed to.  If a project's Javadoc says nothing about nullness,
then you should not assume that the specification forbids null, nor that it
permits it.
(If the project's Javadoc explicitly mentions null everywhere it is permitted,
then you can assume it is forbidden elsewhere, where the author omitted those
statements.)

If a fact is not mentioned in the documentation, then it is usually an
implementation detail.  Clients should not depend on implementation
details, which are prone to change without notice.
(In some cases, you can infer some facts from the
implementation, such as that null is permitted for a method's parameter or
return type if the implementation explicitly passes null to the method or
the method implementation returns null.)

If there is a fact that you think should be in the library's documentation
but was unintentionally omitted by its authors, then please submit a bug
report asking them to update the documentation to reflect this fact.  After
they do, you can also express the fact as an annotation.

% If you wish to depend on your assumption that the behavior will never
% change, then you can create a local stub file as a workaround while you
% wait for the library documentation to be updated.


\subsectionAndLabel{Report bugs upstream}{library-tips-report-bugs}

While annotating the library, you may discover bugs or missing/wrong
documentation.  If you have a documentation improvement or a bug fix, then
open a pull request against the upstream version of the library.  This will
benefit all users of the library.  And, once the documentation is updated,
you will be able to add annotations that are consistent with the
documentation.


\subsectionAndLabel{Fully annotate the library, or indicate which parts you did not}{library-tips-fully-annotate}

If you do not annotate all the files in the library, then use
\refqualclass{framework/qual}{AnnotatedFor} to indicate what files you have
annotated.

Whenever you annotate any part of a file, fully annotate the file!  That
is, write annotations for all the methods and fields, based on their
documentation.  Here are reasons for this rule:

\begin{itemize}
\item
  If you annotate just part of the file, then users may be surprised that
  calls to some methods type-check as expected whereas other methods do not
  (because they have not been annotated).
\item
  Annotating one method or field at a time may lead to inconsistencies
  between different parts of the file.  Different people may make different
  assumptions, might write annotations in a way that is locally convenient
  but globally inconsistent, or might not read all the documentation of the
  class to understand how it works.
\item
  It is not much more effort to annotate an entire class versus one method
  or field.  In either case it is usually necessary to understand the entire
  class's design and implementation.  Once you have done that, you might as
  well annotate the whole thing.
\item
  If you fully annotate the file, it is possible to type-check the library to
  verify the annotations.  (Even if you do not do this right now, it eases
  the task in the future.)
\end{itemize}


\subsectionAndLabel{Verify your annotations}{library-tips-verify}

Ideally, after you annotate a file, you should type-check the file to verify
the correctness and completeness of your annotations.

An alternative is to
only annotate method signatures.  The alternative is quicker but more
error-prone.  There is no difference from the point of view of clients,
who can only see annotations on public methods and fields.  When you
compile the library, the type-checker will probably issue warnings; you can
supply \<-Awarns> so that the
compiler will still produce \<.class> files.


\sectionAndLabel{Creating an annotated library}{annotating-libraries-create}

This section describes how to create an annotated library.

\begin{enumerate}
\item See the
  \ahref{https://search.maven.org/search?q=org.checkerframework.annotatedlib}{the
  \<org.checkerframework.annotatedlib> group in the Central Repository}
  to find out whether an annotated version of the library already exists.
%% This adds clutter to the source code, so omit the step from these instructions.
% \item Optionally, run a script that adds
%     \<\refqualclass{framework/qual}{AnnotatedFor}(\ttlcb\ttrcb)>
%     to each class.  [[TODO: say where to find this script.]]
%     This step has no semantic effect.  It only saves you the trouble of
%     typing those 17 characters later.

  \begin{itemize}
  \item
    If it exists, but you want to add annotations for a different checker:

    Clone its repository from \url{https://github.com/typetools/}, and tweak
    its buildfile to run an additional checker.

  \item
    If it does not already exist:

    Fork the project.  (First, verify that its license permits forking.)

    Add a line in its main README to indicate that this is an annotated
    version of the library.  That line should also indicate how to obtain
    the corresponding upstream version (typically a git tag
    \ahref{https://rawgit.com/typetools/checker-framework/master/docs/developer/developer-manual.html#annotated-library-version-numbers}{corresponding
      to a release}), so that others can see exactly what edits you have
    made.

  Adjust the library's
  build process, such as an Ant, Gradle, or Maven buildfile (for guidance,
  see \chapterpageref{external-tools}).
    Every time the build system runs the compiler, it should:
    \begin{itemize}
    \item
      pass the \<-AuseConservativeDefaultsForUncheckedCode=source,bytecode>
      command-line option and
    \item
      run every pluggable type-checker for which any
      annotations exist, using \<-processor
      \emph{TypeSystem1},\emph{TypeSystem2},\emph{TypeSystem3}>
    \end{itemize}

  You are not adding new build targets, but modifying existing targets.
  The reason to run every type-checker is to verify
  the annotations you wrote, and to use appropriate defaults for all
  unannotated type uses.
  \end{itemize}

\item Annotate some files.
%  You can determine which files need to be annotated by using the
%  \<-AprintUnannotatedMethods> command-line argument while type-checking a
%  client of the library.

  % This is a strong recomendation, but not a requirement.  @AnnotatedFor
  % can be written on a method as well.
  When you annotate a file, annotate the whole thing, not just a few of its
  methods.  Add an
  \<\refqualclass{framework/qual}{AnnotatedFor}(\ttlcb"\emph{checkername}"\ttrcb)>
  annotation to each class declaration, or augment an existing \<@AnnotatedFor>
  annotation.

\item
  Build the library.

  Because of the changes that you made in step 1, this will run pluggable
  type-checkers.  If there are any compiler warnings, fix them and re-compile.

  Now you have a \<.jar> file that you can use while type-checking and at
  run time.

\item
  Tell other people about your work so that they can benefit from it.

  \begin{itemize}
  \item
    Please inform the Checker Framework developers
    about your new annotated library by opening a pull request or an issue.
    This will let us add your annotations to a repository in
    \url{https://github.com/typetools/} and upload a compiled artifact to
    the Maven Central Repository.

  \item
    Optionally, encourage the library's maintainers to accept your annotations into its
    main version control repository.  This suggestion will be most
    compelling if you have already reported bugs in the library that were
    revealed by pluggable type-checking.  Once the annotations are in the
    main version control repository, they will be easier
    to maintain, the library will obtain the correctness guarantees of
    pluggable type-checking, and there will be no need
    to distribute an annotated version of the library.

    If the library maintainers do not accept the annotations, then
    periodically, such as when a new version of the library is released,
    pull changes from upstream (the library's main version control system)
    into your fork, add annotations to any newly-added methods in classes
    that are annotated with \<@AnnotatedFor>, rebuild to create an updated
    \<.jar> file, and inform the Checker Framework developers by opening an
    issue or issuing a pull request.
  \end{itemize}

\end{enumerate}


\sectionAndLabel{Creating an annotated JDK}{annotating-jdk}

When you create a new checker, you need to also supply annotations for
parts of the JDK\@.  You can do so either as stub files or in a copy of the
JDK source code, as described in Section~\ref{creating-a-checker-annotated-jdk}.
Section~\ref{reporting-bugs-annotated-libraries} tells how to improve
JDK annotations for an existing type system.


\sectionAndLabel{Compiling partially-annotated libraries}{compiling-libraries}

If you \emph{completely} annotate a library, then you can compile it using a
pluggable type-checker.  You get a guarantee that the library contains no
errors (that is, it is consistent with the specifications you wrote).

The rest of this section tells you how to compile a library if you
\emph{partially} annotate it:  that is, you write annotations for some of its
classes but not others.
(There is another type of partial annotation, which is when you annotate
method signatures but do not type-check the bodies.
See Section~\ref{library-tips}.)

Here are two concerns you may have when partially annotating a library:

\begin{itemize}
\item
  Ignoring type-checking errors in unannotated parts of the library.
  Use the \<-AskipDefs> or \<-AonlyDefs> command-line arguments; see
  Section~\ref{askipdefs}.

\item
  Putting conservative annotations in unannotated parts of the library.
  The checker needs to use normal defaulting rules
  (Section~\ref{climb-to-top}) for code you have annotated and conservative
  defaulting rules (Section~\ref{defaults-classfile}) for code you have not
  yet annotated.  This section describes how to do this.  You use
  \<@AnnotatedFor> to indicate which classes you have annotated.
\end{itemize}



\subsectionAndLabel{The \<-AuseConservativeDefaultsForUncheckedCode=source,bytecode> command-line argument}{AuseConservativeDefaultsForUncheckedCodesource}

\begin{sloppypar}
When compiling a library that is not fully annotated, use command-line
argument \<-AuseConservativeDefaultsForUncheckedCode=source,bytecode>.  This causes
the checker to behave normally for classes with a relevant \<@AnnotatedFor>
annotation.  For classes without \<@AnnotatedFor>, the checker uses
conservative defaults
(see Section~\ref{defaults-classfile}) for any type use with no explicit
user-written annotation, \emph{and} the checker issues no warnings.
\end{sloppypar}

The \refqualclass{framework/qual}{AnnotatedFor} annotation, written on a
class, indicates that the class has been annotated for certain type
systems.  For example, \<@AnnotatedFor(\ttlcb"nullness", "regex"\ttrcb)> means that
the programmer has written annotations for the Nullness and Regular
Expression type systems.  If one of those two type-checkers is run,
the \<-AuseConservativeDefaultsForUncheckedCode=source,bytecode> command-line argument
has no effect and this class is treated normally:
unannotated types are defaulted using normal source-code
defaults and type-checking warnings are issued.
\refqualclass{framework/qual}{AnnotatedFor}'s arguments are any string that
may be passed to the \<-processor> command-line argument:  the
fully-qualified class name for the checker, or a shorthand for built-in
checkers (see Section~\ref{shorthand-for-checkers}).
Writing \<@AnnotatedFor> on a class doesn't necessarily mean that you wrote
any annotations, but that the user examined the source code and verified
that all appropriate annotations are present.

\begin{sloppypar}
Whenever you compile a class using the Checker Framework, including when
using the \<-AuseConservativeDefaultsForUncheckedCode=source,bytecode> command-line
argument, the resulting \<.class> files are fully-annotated; each type use
in the \<.class> file has an explicit type qualifier for any checker that
is run.
\end{sloppypar}


\sectionAndLabel{Using stub classes}{stub}

A ``stub class'' is Java source code that is allowed to omit certain parts,
such as method bodies.  (See Section~\ref{stub-format} for details
about the stub file format.)  You use a stub file to write annotations
for a library, when you cannot edit and recompile the library.  A
checker uses the annotated signatures at compile time, instead of or in
addition to annotations that appear in the library's \<.class> files.

Section~\ref{annotating-jdk} explains how you should choose between
creating stub classes or creating an annotated library.
Section~\ref{stub-creating} describes how to create stub classes.
Section~\ref{stub-using} describes how to use stub classes.
These sections illustrate stub classes via the example of creating a \refqualclass{checker/interning/qual}{Interned}-annotated
version of \code{java.lang.String}.  You don't need to repeat these steps
to handle \code{java.lang.String} for the Interning Checker,
but you might do something similar for a different class and/or checker.


\subsectionAndLabel{Using a stub file}{stub-using}

The \code{-Astubs} argument causes the Checker Framework to read
annotations from annotated stub classes in preference to the unannotated
original library classes.  For example:

\begin{myxsmall}
\begin{Verbatim}
  javac -processor org.checkerframework.checker.interning.InterningChecker \
    -Astubs=path/to/String.astub:stubs MyFile.java MyOtherFile.java ...
\end{Verbatim}
\end{myxsmall}

Each stub path entry is a stub file (ending with \<.astub>), directory, or
\<.jar> file; specifying a directory or \<.jar> file is
equivalent to specifying every file in it whose name ends with
\code{.astub}.  The stub path entries are delimited by
\<File.pathSeparator> (`\<:>' for Linux and Mac, `\<;>' for Windows).

A checker automatically reads some of its own stub files, even without a
\<-Astubs> command-line argument; see
Section~\ref{creating-a-checker-annotated-jdk}.


\subsectionAndLabel{Multiple specifications for a method}{stub-multiple-specifications}

There are three possible sources of information for a given
element: source files, stub files, and bytecode files.
Usually source files take precedence over the other two,
and stub files take precedence over bytecode.
In other words:
\begin{itemize}
\item
  If file \<A.java> is being compiled, then by default any stub for class
  \<A> is ignored.  See below for how to change this behavior and respect
  both types of annotations.
\item
  An un-annotated type variable in a stub file is used instead of
  annotations on a type variable in bytecode.
  \\
  Use the \<-AstubWarnIfRedundantWithBytecode> command-line option to get a
  warning whenever a stub file specification is redundant with bytecode
  annotations.
  %% Uncomment when https://tinyurl.com/cfissue/2759 is fixed.
  % Use the \<-AstubWarnIfOverwritesBytecode> command-line option to get a
  % warning whenever a stub file overwrites bytecode annotations.
\item
  If a stub file does not mention a method or field (and no source is
  available), its annotations are taken from bytecode.
\end{itemize}

If a method appears in more than one stub file (or twice in the same
stub file), then, for each type hierarchy, the last annotation read is used.

The annotated JDK is read as a stub file.

The command-line option \<-AmergeStubsWithSource> tells the checker to use
both source files and stub files.  The checker permits only values that are
permitted by \emph{both} the source and stub annotations.  (This is called
the greatest lower bound, or GLB, of the types.)

As an example of GLB, suppose the source annotation says the value is in
the range [1..20] and the stub file says the value is in the range
[11..30].  Since both sources of information are trusted, the value must be
in the range [11..20].  Equivalently, the GLB of
\refqualclasswithparams{common/value/qual}{IntRange}{from=1, to=20} and
\refqualclasswithparams{common/value/qual}{IntRange}{from=11, to=30} is
\refqualclasswithparams{common/value/qual}{IntRange}{from=11, to=20}.
As another example, the GLB of
\refqualclasswithparams{checker/nullness/qual}{KeyFor}{\{"map1", "map2"\}} and
\refqualclasswithparams{checker/nullness/qual}{KeyFor}{\{"map2", "map3"\}} is
\refqualclasswithparams{checker/nullness/qual}{KeyFor}{\{"map1", "map2", "map3"\}}
since the value is known to be a key for all three maps.


\subsectionAndLabel{Stub file format}{stub-format}

Every Java file is a valid stub file.  However, you can omit information
that is not relevant to pluggable type-checking; this makes the stub file
smaller and easier for people to read and write.
Also note that the stub file's extension must be \<.astub>, not \<.java>.

As an illustration, a stub file for the Interning type system
(Chapter~\ref{interning-checker}) could be:

\begin{Verbatim}
  import org.checkerframework.checker.interning.qual.Interned;
  package java.lang;
  @Interned class Class<T> {}
  class String {
    @Interned String intern();
  }
\end{Verbatim}

The stub file format is allowed to differ from Java source code in the
following ways:
\begin{description}

\item{\textbf{Method bodies:}}
  The stub class does not require method bodies for classes; any method
  body may be replaced by a semicolon (\code{;}), as in an interface or
  abstract method declaration.

\item{\textbf{Method declarations:}}
  You only have to specify the methods that you need to annotate.
  Any method declaration may be omitted, in which case the checker reads
  its annotations from library's \<.class> files.  (If you are using a stub class, then
  typically the library is unannotated.)

\item{\textbf{Declaration specifiers:}}
  Declaration specifiers (e.g., \<public>, \<final>, \<volatile>)
  may be omitted.

\item{\textbf{Return types:}}
  The return type of a method does not need to match the real method.
  In particular, it is valid to use \<java.lang.Object> for every method.
  This simplifies the creation of stub files.

\item{\textbf{Import statements:}}
  Imports may appear at the beginning of the file or after any package declaration.
  The only required import statements are the ones to import type
  annotations.  Import statements for types are optional.

\item{\textbf{Multiple classes and packages:}}
  The stub file format permits having multiple classes and packages.
  The packages are separated by a package statement:
  \<package my.package;>.  Each package declaration may occur only once; in
  other words, all classes from a package must appear together.

\end{description}



\subsectionAndLabel{Creating a stub file}{stub-creating}


\subsubsectionAndLabel{If you have access to the Java source code}{stub-creating-with-source}

Every Java file is a stub file.  If you have access to the Java file,
copy file \<A.java> to \<A.astub>.  You can add
annotations to the signatures, leaving the method bodies unchanged.
The stub file parser silently ignores any annotations that it cannot
resolve to a type, so don't forget the \<import> statement.

Optionally (but highly recommended!), run the type-checker to verify that
your annotations are correct.  When you run the type-checker on your
annotations, there should not be any stub file that also contains
annotations for the class.  In particular, if you are type-checking the JDK
itself, then you should use the \<-Aignorejdkastub> command-line option.

This approach retains the original
documentation and source code, making it easier for a programmer to
double-check the annotations.  It also enables creation of diffs, easing
the process of upgrading when a library adds new methods.  And, the
annotations are in a format that the library maintainers can even
incorporate.

The downside of this approach is that the stub files are larger.  This can
slow down the Checker Framework, because it parses the stub files each time
it runs.
% Furthermore, a programmer must search the stub file
% for a given method rather than just skimming a few pages of method signatures.

Alternatively, you can minimize source files to make them more suitable as stub files.
Use the \<JavaStubifier> to convert, in-place, all \<.java> files in given directories into
minimal stub files.

\begin{Verbatim}
  mkdir project-stubs
  cp -R project/src project-stubs
  java -cp $CHECKERFRAMEWORK/checker/dist/checker.jar org.checkerframework.framework.stub.JavaStubifier project-stubs
  find project-stubs -type f -name "*.java" -exec rename 's/.java$/.astub/' {} \;
\end{Verbatim}

Supply it a list of directories to process. Replacement happens in-place, so make sure to
process a copy of your sources.

You can now provide \<project-stubs> as a stub directory using \<-Astubs=project-stubs> as
an additional command-line option.

If you wish to use a single stub file, rather than a stub directory,
you can concatenate all the minimized \<.java> files into a single \<.astub> file:

\begin{Verbatim}
find project-stubs/ -name "*.java" -type f | xargs cat > project.astub
\end{Verbatim}

% That file will contain many non-unique import statements, but that shouldn't be harmful.


\subsubsectionAndLabel{If you do not have access to the Java source code}{stub-creating-without-source}

If you do not have access to the library source code, then you can create a
stub file from the class file (Section~\ref{stub-creating}),
and then annotate it.  The rest of this section describes this approach.


\begin{enumerate}

\item
  Create a stub file by running the stub class generator.  (\<checker.jar> must be on your classpath.)

\begin{Verbatim}
  cd nullness-stub
  java -cp $CHECKERFRAMEWORK/checker/dist/checker.jar \
    org.checkerframework.framework.stub.StubGenerator java.lang.String > String.astub
\end{Verbatim}

  Supply it with the fully-qualified name of the class for which you wish to
  generate a stub class.  The stub class generator prints the
  stub class to standard out, so you may wish to redirect its output to a
  file.

\item
  Add import statements for the annotations.  So you would need to
add the following import statement at the beginning of the file:

\begin{Verbatim}
  import org.checkerframework.checker.interning.qual.*;
\end{Verbatim}

\noindent
The stub file parser silently ignores any annotations that it cannot
resolve to a type, so don't forget the import statement.
Use the \<-AstubWarnIfNotFound> command-line option to see warnings
if an entry could not be found.

\item
  Add annotations to the stub class.  For example, you might annotate
  the \sunjavadoc{java.base/java/lang/String.html\#intern()}{String.intern()} method as follows:

\begin{Verbatim}
  @Interned String intern();
\end{Verbatim}

  You may also remove irrelevant parts of the stub file; see
  Section~\ref{stub-format}.

\end{enumerate}


\subsectionAndLabel{Distributing stub files}{stub-distributing}

If you are writing stub files but are not writing a checker, you can place
the stub files anywhere that is convenient.  However, please consider
providing your stub files to the checker author, so that other users can
also benefit from the stub files you wrote.

If you are distributing stub files with a checker implementation, you must
configure your build system to copy the stub files into the distributed jar for
your checker.  Here are instructions if you are using the Checker Framework
Gradle plugin:
\begin{itemize}
\item
  If the stub files appear in the same directory as the checker class, which is
  a subtype of \<BaseTypeChecker>, use the following build configuration:
\begin{Verbatim}
  sourceSets {
    main {
        resources {
            srcDirs += ['src/main/java']
        }
    }
  }
\end{Verbatim}
\item
  If the stub files appear under the \<src/main/resources/>
  directory in a sub-directory matching the package of your checker class,
  the default Gradle build configuration is sufficient.
\end{itemize}
In either case, recall that a \refqualclass{framework/qual}{StubFiles}
annotation on the checker class lists stub files that are always used.  For
stub files whose use is optional (for example, because the behavior is
unsound, or unsound except in certain circumstances), users must supply the
\<-Astubs=...> command-line option.


If a stub file contains annotations that are used by the framework rather
than by any specific checker (such as purity annotations), and you wish to
distribute it with the Checker Framework, put the stub file in directory
\<checker/resources/>.  You can also do this if the stub file has
annotations for multiple checkers.  To use a stub file in directory
\<checker/resources/>, users must supply the
\<-Astubs=checker.jar/stub-file-name.astub> command-line option.


\subsectionAndLabel{Troubleshooting stub libraries}{stub-troubleshooting}


\subsubsectionAndLabel{Type-checking does not yield the expected results}{stub-troubleshooting-type-checking-results}

By default, the stub parser silently ignores
annotations on unknown classes and methods.
The stub parser also silently ignores unknown annotations, so don't forget to
\<import> any annotations.
Some command-line options make the stub parser issue more warnings:

\begin{description}
\item[\<-AstubWarnIfNotFound>]
  Warn whenever some element of a stub file cannot be found.
  The \<@NoStubParserWarning> annotation on a package or type in a stub file
  overrides the \<-AstubWarnIfNotFound> command-line option, and no warning
  will be issued.

\item[\<-AstubWarnIfNotFoundIgnoresClasses>]
  Modifies the behavior of \<-AstubWarnIfNotFound>
  to report only missing methods/fields, but ignore missing classes, even if
  other classes from the same package are present.
  Useful if a package spans more than one jar.

%% Uncomment when https://tinyurl.com/cfissue/2759 is fixed.
% \item[\<-AstubWarnIfOverwritesBytecode>]
%   Warn whenever some element of a
%   stub file overwrites annotations contained in bytecode.
\end{description}

Finally,
use command-line option {\bf\<-AstubDebug>} to output debugging messages while
parsing stub files, including about unknown classes, methods, and
annotations.  This overrides the \<@NoAnnotationFileParserWarning> annotation.



\subsubsectionAndLabel{Problems parsing stub libraries}{stub-troubleshooting-parsing}

When using command-line option \<-AstubWarnIfNotFound>,
an error is issued if a stub file has a typo or the API method does not
exist.

Fix an error of the form
\begin{Verbatim}
AnnotationFileParser: Method isLLowerCase(char) not found in type java.lang.Character
\end{Verbatim}

\noindent
by removing the extra ``L'' in the method name.

Fix an error of the form
\begin{Verbatim}
AnnotationFileParser: Method enableForegroundNdefPush(Activity,NdefPushCallback)
      not found in type android.nfc.NfcAdapter
\end{Verbatim}

\noindent
by removing the method \<enableForgroundNdefPush(...)> from
the stub file, because it is not defined in class \<android.nfc.NfcAdapter>
in the version of the library you are using.


\sectionAndLabel{Ajava Files}{ajava-files}

Like stub files, ajava files store annotations that can be read by the
Checker Framework.  Ajava files are mostly used as
the output of whole program inference and not for annotating libraries.


\subsectionAndLabel{Using an Ajava File}{ajava-using}

The \code{-Aajava} command argument works like \code{-Astubs} (see
Section~\ref{stub-using}). It takes a colon-separated list of
files and directories containing ajava files, which end in the \code{.ajava}
extension.

For example:

\begin{myxsmall}
\begin{Verbatim}
  javac -processor org.checkerframework.checker.interning.InterningChecker \
    -Aajava=path/to/String.astub:stubs MyFile.java MyOtherFile.java ...
\end{Verbatim}
\end{myxsmall}


\subsectionAndLabel{Contents of an Ajava file}{ajava-contents}

An ajava file must contain valid Java source code. Annotations in that code are
read. When reading an ajava file for a file that also exists as source code, the
ajava file and source file must match exactly with the following exceptions:
\begin{itemize}
  \item The two may differ in annotations.
  \item The ajava file may have explicit receivers where the source file
    doesn't. If a source file has a method declaration \code{myMethod()},
    the ajava file might contain \code{myMethod(MyClass this)}.
  \item The ajava file may have explicit type bounds that are implicit in the source file.
    If a source file has a type argument \<?>,
    the ajava file might contain \code{? extends Object}.
\end{itemize}


\subsectionAndLabel{Annotations in Ajava Files}{ajava-annotations}

Annotations on method parameters, method returns, and fields are read (this
is like stub files).
If there's an annotation in both an ajava file and
a source file, the greatest lower bound is used (as with the
\code{-AmergeStubsWithSource} option).

Ajava files may contain annotations within anonymous classes and local classes.
Annotations in such locations are ignored in stub files.x


\sectionAndLabel{Troubleshooting/debugging annotated libraries}{libraries-troubleshooting}

Sometimes, it may seem that a checker is treating a library as unannotated
even though the library has annotations.  The compiler has two flags that
may help you in determining whether library files are read, and if they are
read whether the library's annotations are parsed.

\begin{description}
\item \<-verbose>
  Outputs info about compile phases --- when the compiler
  reads/parses/attributes/writes any file.  Also outputs the classpath and
  sourcepath paths.
\item \<-XDTA:parser> (which is equivalent to \<-XDTA:reader> plus \<-XDTA:writer>)
  Sets the internal \<debugJSR308> flag, which outputs information about
  reading and writing.
\end{description}

A syntax error in a stub file or the annotated JDK can lead to the file
being silently ignored.  A typo in an annotation name, or a missing
\<import> statement, can lead to annotations being silently ignored.


% LocalWords: plugin utils util dist RuntimeException NonNull TODO AFU enum
% LocalWords: sourcepath Nullness javac classpath src quals pathSeparator JDKs
% LocalWords: jdk Astubs skipUses astub AskipUses toArray JDK6 xvzf javax
% LocalWords: CollectionToArrayHeuristics BaseTypeVisitor Xbootclasspath
% LocalWords: Interning's UsesObjectEquals ApermitMissingJdk AonlyUses java pre
% LocalWords: Aignorejdkastub AstubWarnIfNotFound AstubDebug dont local'
% LocalWords: enableForgroundNdefPush XDTA debugJSR308 BCEL getopt jdk8
% LocalWords: NoAnnotationFileParserWarning CHECKERFRAMEWORK AnnotatedFor regex
% LocalWords: AuseConservativeDefaultsForUnannotatedCode buildfile qual
% LocalWords: AprintUnannotatedMethods checkername AskipDefs bcel mkdir
% LocalWords: AuseSafeDefaultsForUnannotatedSourceCode TypeSystem1 cd
% LocalWords: TypeSystem2 TypeSystem3 AuseConservativeDefaultsForUncheckedCode ln
% LocalWords: mychecker DIRS README TypeSystem un debugJSR org boolean
% LocalWords: AstubWarnIfOverwritesBytecode Awarns AssumeAssertion
% LocalWords: Makefile PuseLocalJdk jdkShaHash AonlyDefs
% LocalWords: AuseConservativeDefaultsForUncheckedCodesource
% LocalWords: AstubWarnIfNotFoundIgnoresClasses
