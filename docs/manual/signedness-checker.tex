\htmlhr
\chapterAndLabel{Signedness Checker}{signedness-checker}

The Signedness Checker guarantees that signed and unsigned values are not mixed
together in a computation. In addition, it prohibits meaningless operations, such
as division on an unsigned value.


\sectionAndLabel{Annotations}{signedness-checker-annotations}

The Signedness Checker uses type annotations to indicate the signedness that the programmer intends an expression to have.

\begin{figure}
\includeimage{signedness}{3.5cm}
\caption{The type qualifier hierarchy of the signedness annotations.
Qualifiers in gray are used internally by the type system but should never be written by a programmer.}
\label{fig-signedness-hierarchy}
\end{figure}

These are the qualifiers in the signedness type system:

\begin{description}

\item[\refqualclass{checker/signedness/qual}{Unsigned}]
    indicates that the programmer intends the value to be
    interpreted as unsigned.
    That is, if the most significant bit in the bitwise representation is
    set, then the bits should be interpreted as a large positive value.

\item[\refqualclass{checker/signedness/qual}{Signed}]
    indicates that the programmer intends the value to be
    interpreted as signed.
    That is, if the most significant bit in the bitwise representation is
    set, then the bits should be interpreted as a negative value.
    This is the default annotation.

\item[\refqualclass{checker/signedness/qual}{Constant}]
    indicates that a value is a compile-time constant and could be
    interpreted as unsigned or signed.
    This annotation is used internally, and should not
    be written by the programmer.

\item[\refqualclass{checker/signedness/qual}{UnknownSignedness}]
    indicates that a value's type is not relevant or known to this checker.
    This annotation is used internally, and should not be
    written by the programmer.

\item[\refqualclass{checker/signedness/qual}{SignednessBottom}]
  indicates that the value is \<null>.
    This annotation is used internally, and should not
    be written by the programmer.

\end{description}

Signedness is primarily about how the bits of the representation are
interpreted, not about the values that it can represent.  An unsigned value
is always positive, but just because a variable's value is positive does
not mean that it should be marked as \<@Unsigned>.  If variable $v$ will be
compared to a signed value, or used in arithmetic operations with a signed
value, then $v$ should have signed type.


\subsectionAndLabel{Default qualifiers}{signedness-checker-annotations-default-qualifiers}

The only type qualifier that the programmer should need to write is
\refqualclass{checker/signedness/qual}{Unsigned}.
When a programmer leaves an expression unannotated, the
Signedness Checker treats it in one of the following ways:

\begin{itemize}

    \item
    All \code{byte}, \code{short}, \code{int}, and \code{long} literals default
    to \refqualclass{checker/signedness/qual}{Constant}.
    \item
    All \code{byte}, \code{short}, \code{int}, and \code{long} variables default
    to \refqualclass{checker/signedness/qual}{Signed}.
    \item
    All other expressions default to \refqualclass{checker/signedness/qual}{UnknownSignedness}.

\end{itemize}


\sectionAndLabel{Prohibited operations}{signedness-checker-prohibited-operations}

The Signedness Checker prohibits the following uses of operators:

\begin{itemize}

    \item
    Division (\code{/}) or modulus (\code{\%}) with an \code{@Unsigned}
    operand.
    \item
    Signed right shift (\verb|>>|) with an \code{@Unsigned} left operand.
    \item
    Unsigned right shift (\verb|>>>|) with a \code{@Signed} left operand.
    \item
    Greater/less than (or equal) comparators
    (\code{<}, \code{<=}, \code{>}, \code{>=}) with an \code{@Unsigned}
    operand.
    \item
    Any other binary operator with one \code{@Unsigned} operand and one
    \code{@Signed} operand, with the exception of left shift (\verb|<<|).

\end{itemize}

Like every type-checker built with the Checker Framework, the Signedness
Checker ensures that assignments and pseudo-assignments have consistent types.
For example, it is not permitted to assign a \code{@Signed} expression to an
\code{@Unsigned} variable or vice versa.


\sectionAndLabel{Rationale}{signedness-checker-rationale}

The Signedness Checker prevents misuse of unsigned values in Java code.
Most Java operations interpret operands as signed.  If applied to unsigned
values, those operations would produce unexpected, incorrect results.

Consider the following Java code:

\begin{Verbatim}
  int s1 = -2;
  int s2 = -1;

  @Unsigned int u1 = 0xFFFFFFFE;  // unsigned: 2^32 - 2, signed: -2
  @Unsigned int u2 = 0xFFFFFFFF;  // unsigned: 2^32 - 1, signed: -1

  int result;

  result = s1 / s2;   // OK: result is 2, which is correct for -2 / -1
  result = u1 / u2;   // ERROR: result is 2, which is incorrect for (2^32 - 2) / (2^32 - 1)

  int s3 = -1;
  int s4 = 5;

  @Unsigned int u3 = 0xFFFFFFFF;    // unsigned: 2^32 - 1, signed: -1
  @Unsigned int u4 = 5;

  result = s3 % s4;    // OK: result is -1, which is correct for -1 % 5
  result = u3 % u4;    // ERROR: result is -1, which is incorrect for (2^32 - 1) % 5
\end{Verbatim}

These examples illustrate why division and modulus with an unsigned operand
are illegal.  Other uses of operators are prohibited for similar reasons.


\sectionAndLabel{Utility routines for manipulating unsigned values}{signedness-utilities}

Class \refclass{checker/signedness}{SignednessUtil} provides static
utility methods for working with unsigned values.  Some of these
re-implement functionality in JDK 8, making it available in earlier
versions of Java.  Others provide new functionality.  All of them are
properly annotated with \refqualclass{checker/signedness/qual}{Unsigned}
where appropriate, so using them may reduce the number of annotations that
you need to write.

Class \refclass{checker/signedness}{SignednessUtilExtra} contains more utility
methods that reference packages not included in Android.  This class is not
included in \code{checker-qual.jar}, so you may want to copy the methods to your code.
